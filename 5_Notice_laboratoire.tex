Ce système doit pouvoir être exploré par des étudiants lors d'un cours au sein du laboratoire COMATEC-LANS. Étant donné que le manuel d'utilisation du système fourni par Thorlabs comprends 108 pages au total~\cite{manualPortableOpticalTweezers}, il semble nécessaire de concevoir une notice de laboratoire plus compacte, afin de faciliter sa prise en main. La notice complète se trouve en annexes à la page~\pageref{annexe:notice_labo}. Ci-dessous, une liste résumée du contenu de cette notice.
\section{Contenu de la notice}
Cette notice de laboratoire est structurée de manière à guider pas à pas l'utilisateur qui souhaite découvrir ce système de pinces optiques. Elle est organisée selon les phases suivantes :

\begin{enumerate}
    \item \textbf{Présentation des composants du système} : description détaillée des éléments constituant le système, avec un focus particulier sur les dispositifs de sécurité pour que l'utilisateur puisse facilement les identifier et comprendre leur rôle.

    \item \textbf{Installation et prise en main des logiciels} : guide pas-à-pas pour l'installation des logiciels ThorCam et Kinesis, accompagnée d'une explication claire de leur rôle dans la capture d'image et le contrôle des moteurs.

    \item \textbf{Préparation de l'échantillon} : protocole pour préparer correctement l'échantillon, afin de garantir une manipulation optimale par la pince optique.

    \item \textbf{Réglage précis de la hauteur de l'échantillon} : méthodes pour ajuster la position verticale de la lame dans le système, afin d'optimiser le focus du laser et assurer un piégeage efficace des particules.

    \item \textbf{Manipulation expérimentale des billes} : réalisation d'expériences de déplacement des particules à l'aide de la pince optique, avec un calcul détaillé permettant d'estimer la force maximale exercée par le laser pour maintenir les billes piégées.

    \item \textbf{Remise en ordre du système} : Explications pour éteindre, nettoyer et ranger le matériel en fin d'utilisation.
\end{enumerate}