L'utilisation de lasers puissants dans un système expérimental doit être une des premières préoccupations de la personne responsable.

Dans le cadre de ce travail de bachelor, deux niveaux de protection ont été élaborés afin de garantir au mieux la sécurité du laser :

\begin{enumerate}
    \item Le premier niveau est électrique et repose sur deux dispositifs :
          \begin{itemize}
              \item Le système intégré interlock du driver du laser va être modifié pour fonctionner avec les protections mécaniques.
              \item Un boîtier contenant un bouton d'arrêt d'urgence ainsi qu'une clé de maintenance vont être apportés en plus dans le système interlock.
          \end{itemize}
    \item Le second niveau est mécanique et comprend deux mécanismes :
          \begin{itemize}
              \item La première est située à l'entrée du faisceau laser.
              \item La seconde est située vers le microscope, en fin de trajet du laser.
          \end{itemize}
\end{enumerate}

Le chapitre \ref{chapter:securite_electrique} va expliquer la partie électrique, tandis que le chapitre \ref{chapter:securite_mecanique} s'occupera de la partie mécanique.