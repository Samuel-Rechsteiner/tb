\section{Première version}
Une étape essentielle dans tout système électrique réside premièrement dans son câblage. Ci-dessous, une photo du câblage avec des annotations pour une bonne compréhension du travail réalisé (voir Figure \ref{cablage_kit_annoté}). Le but a été d'avoir un câblage le plus propre et pratique possible.

\newpage
\begin{figure}[H]
    \begin{center}
        \includegraphics[width=0.9\textwidth]{assets/figures/Cablage_du_kit/Cablage_annote.png}
    \end{center}
    \caption{Câblage du kit avec annotations}
    \label{cablage_kit_annoté}
\end{figure}

Du point de vue de la propreté, on peut apercevoir, en \textcolor[RGB]{230, 230, 0}{jaune}, que des goulottes ont été installées.

Les encadrés en \textcolor{red}{rouge} sont les 4 alimentations. Il y'a une alimentation pour :
\begin{itemize}
    \item Les 2 drivers des servomoteurs.
    \item Le driver du laser.
    \item Le driver de la LED.
\end{itemize}

En complément des alimentations, les encadrés en \textcolor[RGB]{0, 201, 18}{vert} sont les câbles permettant de communiquer avec les différents composants. La liste des câbles pour la communication ci-dessous :
\begin{itemize}
    \item 1 câble pour chaque driver de servomoteur, 2 au total.
    \item 1 câble pour le driver du laser.
    \item 1 câble pour la caméra.
\end{itemize}