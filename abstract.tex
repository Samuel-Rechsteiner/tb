% Francais
Les pinces optiques sont des dispositifs basés sur l'utilisation d'un faisceau laser focalisé permettant de piéger et manipuler des objets microscopiques, tels que des microbilles. Le système exploite la force de gradient générée par la focalisation de la lumière, qui attire les particules au centre du faisceau laser.

Ce projet de Bachelor consiste à mettre en service un système de pinces optiques, à transformer un système de sécurité laser de classe 3B (port de lunettes de protection obligatoire) en un système sécurisé de classe 1 (aucune protection requise), pour l'utilisation en toute sécurité par des utilisateur\(\cdot\)trices non-expert\(\cdot\)es. Le projet inclut également la création d'une notice de laboratoire, proposant un manuel d'utilisation simple et clair, ainsi que des expériences réalisables avec le système. Pour faciliter l'utilisation du système, une application regroupant toutes les fonctionnalités nécessaires pour le pilotage de l'ensemble du dispositif est également développée.

Le projet a été réalisé au Laboratoire d'Applications des NanoSciences (COMATECLANS) à la HEIG-VD. La sécurisation du dispositif a été effectuée en intégrant des protections mécaniques, un boîtier avec bouton d'arrêt d'urgence, ainsi que des capteurs électriques associés aux protections: lorsqu'elles sont ouvertes, ces capteurs interrompent automatiquement l'alimentation du laser. Un bouton à clé pour la maintenance a également été câblé. Lorsqu'il est activé, le système optique peut être ajusté avec le laser allumé (le port des lunettes de protection est obligatoire dans ce cas).

En conclusion, le système de pinces optiques a été pris en main, sécurisé et rendu accessible pour des utilisateur\(\cdot\)trices non-expert\(\cdot\)es, pour effectuer des expériences, comme le micropositionnement des microbilles et la mesure de la force du piégeage. Une documentation claire a été créée pour accompagner les utilisateur\(\cdot\)trices, et une interface logicielle intuitive permet de piloter le système de manière simple et complète.

% \asterism

% English
% Optical tweezers are devices based on the use of a focused laser beam to trap and manipulate microscopic objects, such as microbeads. The system exploits the gradient force generated by focusing the light, which attracts the particles to the centre of the laser.

% This Bachelor's project involves commissioning a system of optical tweezers, upgrading it from a class 3B laser safety system (protective glasses must be worn) to a class 1 safety system (no protection required), so that it can be used safely by non-expert users. The project also includes the creation of a simple and clear laboratory manual, proposing experiments to be carried out with the system, as well as the development of an application bringing together all the functions needed to control and operate the entire device.

% The project was carried out at the NanoSciences Applications Laboratory (COMATEC-LANS) at HEIG-VD. The device was made safe by adding mechanical protections, a box with an emergency stop button, and electrical sensors associated with the protections: when they are open, these sensors automatically interrupt the laser's power supply. A key-operated maintenance button has also been wired in. When activated, it enables adjustments to be made to the various components while the laser is still switched on (protective goggles must be worn).

% In conclusion, the optical tweezers system was taken in hand, secured and made accessible for use by non-expert users. Clear documentation has been created to assist users, and an intuitive software interface makes it simple and comprehensive to control the system.